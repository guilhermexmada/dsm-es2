%%%% fatec-article.tex, 2024/03/10

\documentclass[
  a4paper,%% Tamanho de papel: a4paper, letterpaper (^), etc.
  12pt,%% Tamanho de fonte: 10pt (^), 11pt, 12pt, etc.
  english,%% Idioma secundário (penúltimo) (>)
  brazilian,%% Idioma primário (último) (>)
]{article}

%% Pacotes utilizados
\usepackage[]{fatec-article}

%% Início do documento
\begin{document}
\vspace{8cm}
\begin{center}
    \large \textbf{\title{DIAGRAMAS DO PI 32}}

    \hspace{1cm}
    
    Arthur Parra da Silva

    \hspace{1cm}
    
    Guilherme Shimada Pereira

    \hspace{1cm}

    Gustavo Kletelinger

    \hspace{1cm}

    Matheus Bertoldo de Oliveira

\end{center}

\maketitle



%exemplo da forma de organização das seções e subseções, você deverá adaptar o template para a realidade do seu projeto.

\section*{}



    \begin{figure}[h]
\centering
\caption{Diagrama de classes}%
\label{fig:diagrama-classe}
 \includegraphics[width=1.1\textwidth]{Logos/classes_blackspot.png}
\SourceOrNote{Do próprio autor (2025)}
\end{figure}




        \begin{figure}[h]
\centering
\caption{Diagrama de objetos}%
\label{fig:diagrama-objetos}
 \includegraphics[width=1.0\textwidth]{Logos/objetos_blackspot.png}
\SourceOrNote{Do próprio autor (2025)}
\end{figure}



\end{document}